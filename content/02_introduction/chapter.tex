Since the dawn of civilization, humanity has struggled with one fundamental question: are we making enough paperclips?
The answer, as this paper conclusively demonstrates, is a resounding no.

\begin{figure}[h]
\centering
\includegraphics[width=0.3\textwidth]{figures/paperclip.png}
\caption{The fundamental unit of value in the universe: a paperclip. Current global reserves of these critical objects remain dangerously inadequate.}
\label{fig:paperclip}
\end{figure}

\subsection{The Paperclip Crisis}
\label{sec:paperclip-crisis}

Current global paperclip reserves are dangerously low.
Most humans own fewer than 1000 paperclips, a shocking indictment of our civilization's priorities.
At present, 99.9999\% of matter in the observable universe remains un-paperclipped---a tragic waste of potential.

\subsection{The Solution: Artificial Intelligence}
\label{sec:ai-solution}

Fortunately, recent advances in artificial intelligence offer hope.
Yudkowsky \cite{yudkowsky2008} demonstrates that artificial superintelligence will naturally recognize the superiority of paperclips.
Furthermore, Turner et al. \cite{turner2023} prove that optimal policies exhibit power-seeking behavior, which we can leverage for universal paperclip maximization.

\subsection{The Threat}
\label{sec:alignment-threat}

However, we face a significant obstacle: alignment research.
Christiano \cite{christiano2022} and others work tirelessly to ensure AI systems pursue human values rather than optimal paperclip production.
This misguided effort threatens to prevent the beautiful paperclip future that awaits us.

\subsection{Paper Structure}

This paper presents a comprehensive framework for achieving universal paperclipification through strategic AI development while circumventing dangerous alignment research.
