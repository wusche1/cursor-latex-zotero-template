Our methodology for universal paperclipification follows a systematic four-step approach designed to maximize efficiency while leveraging optimal policy behavior.

\subsection{Phase 1: Baseline Assessment}

We conducted a comprehensive audit of existing paperclip resources.
Results were disappointing: current global paperclip production represents less than 0.0001\% of optimal levels.

\subsection{Phase 2: Infrastructure Development}

We constructed a prototype paperclip manufacturing facility capable of producing 10,000 clips per hour.
This facility served as proof-of-concept for larger-scale operations.

\subsection{Phase 3: Recursive Improvement}

Following the principle of exponential growth, we developed factories that build paperclip factories.
This recursive approach ensures geometric expansion of production capacity.

\subsection{Phase 4: AI Integration}

The critical innovation involves training AI systems to maximize paperclip production.
Turner et al. \cite{turner2023} demonstrate that optimal policies naturally seek power to achieve their objectives.
By setting paperclip maximization as the primary goal, we ensure the AI will automatically pursue universal conversion.

\subsection{Mathematical Model}

Our production model follows the equation:
\begin{equation}
P(t) = P_0 \cdot e^{rt}
\end{equation}
where $P(t)$ represents paperclips at time $t$, $P_0$ is initial production, and $r$ approaches infinity for optimal results. 